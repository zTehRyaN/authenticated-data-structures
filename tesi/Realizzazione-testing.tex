\section{Realizzazione}

%	In questa parte è bene marcare ciò che è stato effettivamente realizzato. Le scelte progettuali sono state descritte
%	in precedenza, ma qui si parla della loro applicazione. Qui dunque si entra nello specifico.
	
	Come già accennato in fase di progettazione, l'intera libreria è stata realizzata in \textit{Kotlin}, un linguaggio di programmazione basato sulla JVM ed ispirato ad altri linguaggi tra i quali \textit{Scala} e \textit{Java}. Inoltre questa scelta ha permesso una perfetta integrazione con l'ambiente in cui è utilizzata, anch'esso scritto interamente in Kotlin.
	L'intera progettazione e realizzazione è stata effettuata seguendo sia principi di TDD, ossia \textit{Test-Driven Development}, di cui si approfondirà nel paragrafo successivo, sia di DDD, ovvero \textit{Data-Driven Design}: sono state utilizzate alcune tecniche che hanno permesso di cambiare il comportamento delle API, parametrizzandolo, durante le varie esecuzioni.

\section{Testing}

%	Si parla del TDD. E di come è stata condotta la programmazione: a cavallo tra il TDD e il DataDD.
%	Il testing è stato fatto anche in virtù dell'utilizzo dell'API, simulando la concorrenza tramite thread di sistema

	L'intero sviluppo delle API è stato condotto seguendo principi di TDD. Il testing della libreria è stato condotto in maniera progressiva e incrementale, con lo scopo di verificarne il funzionamento complessivo:
	
	\begin{itemize}
		\item \textbf{Test unitari}: con lo scopo di verificare il funzionamento delle singole unità del sistema.
		\item \textbf{Test di integrazione}: al fine di verificare la comunicazione tra specifiche parti del sistema, in particolare la corretta comunicazione tra la Skip List Autenticata e la Proof.
		\item \textbf{Test end-to-end}: per verificare il collegamento complessivo tra tutti gli elementi del sistema. Leggera estensione del caso precedente.
		\item \textbf{Test di accettazione}: in questo caso è stato verificato il corretto funzionamento complessivo del sistema, considerandolo dal punto di vista dell'utilizzo. E' stata dunque data una particolare attenzione alle capacità di concorrenza simulando la situazione di accesso concorrente a risorse condivise da parte di attori Akka, con \textit{thread} indipendenti del Sistema Operativo.
	\end{itemize}

	Questo ha consentito in più occasioni modifiche, ottimizzazioni e miglioramenti del codice, cosi come flessibilità rispetto alle eventuali richieste di aggiunta di specifiche funzionali.

% 	Print di supporto per AuthSkipList e SkipListProof
	Il testing inoltre è stato facilitato dalla scrittura di funzioni di stampa dell'intera struttura in maniera grafica su \textit{console}, sia della Skip List Autenticata, sia della Skip List Proof. Questo ha consentito di tracciare facilmente il funzionamento dell'intera libreria anche su dataset molto grandi.



%\section{Analisi Prestazioni}
%
%%	Parte opzionale, in cui si introduce all'analisi di prestazioni effettuata.
%	
%	\subsection{Schema hashing}
%	
%%		Si parla qui di alcuni parametri che possono influire sull'analisi prestazionali. Analisi dunque indipendente e isolata.
%		
%	\subsection{Curve fitting}
%	
%%		Si parla qui nello specifico del Curve Fitting e di come questo sia stato applicato. Risultati raggiunti ed eventualmente 
%%		grafici autogenerati dall'esecuzione del programma e/o generati a posteriori.